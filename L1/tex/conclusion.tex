\section{Discussion}
\label{sec:conclusion}

While Figure \ref{fig:dynamics} shows that there is high agreement between the numerical and analytical solutions, this result is dependent on the size of the sponge used for the periodic boundary conditions. In essence, some tuning is necessary to achieve this result, and should be taken into account. Another question may be why the final state is not that of a flat ocean surface, and this would indeed be the cause in a non-rotating system where no force can balance the pressure gradients. In a rotating system, however, this is achieved by the coriolis force.

Figures \ref{fig:energetics_d} and \ref{fig:energetics_f} illustrates that the planetary vorticity and depth have reverse effects on the result, which is not surprising due to their competing dependency in the Rossby radius of deformation. Increasing the depth or reducing the planetary vorticity will increase $R$, which allows for greater kinetic energy retention due to a larger area of geostrophy where energy do not radiate away as gravity waves. The increased fluctuations due to a lower $R$ may be due to more gravity waves oscillating back and forth in the system.

\subsection{Conclusion}
In this report we have studied the dynamics and energetics of geostrophic adjustment by solving the shallow water equations numerically and analytically for a step-wise sea surface height perturbation. We found general good agreement between numerical and analytical solutions. Exploring the effect of changing the Rossby radius, we found that a larger Rossby radius tends to cause a greater fraction of the total energy to be potential energy. This was due to a overall larger contribution from rotational effects balancing pressure gradients.