\section{Theory}
\label{sec:theory}

The linearized shallow water equations are as follows:
	\begin{align}
		&\partial_t u - f_0 v = -g \partial_x h \label{eq:dudt} \\
		&\partial_t v + f_0 u = -g \partial_y h \label{eq:dvdt} \\
		&\partial_t h + D_0 \qty(\partial_x u + \partial_y v) = 0, \label{eq:dhdt}
	\end{align}
where $\partial_{q_i}$ denotes partial differentiation or ordinary differentiation given the context. $u$ and $v$ are the zonal and meridional velocities, respectively, while $h$ is the sea-surface height. The planetary vorticity is $f_0$, while $g$ is the gravitational acceleration, and $D_0$ is the depth of the system. 

In this report, we will study the case of a stepwise sea surface height perturbation of $\pm h_0$
	\begin{equation}
		h_i = h_0
			\begin{cases}
			h_0, & x > 0 \\
			-h_0, & x < 0
			\end{cases},
	\label{eq:initial_h}
	\end{equation}
which can be solved analytically.

\subsection{Dynamics}
The resulting sea surface height after geostrophic adjustment is
	\begin{equation}
		h_f = h_0
			\begin{cases}
				-1 + \exp(-x / R), & x > 0\\
				1 - \exp(x / R), & x < 0
			\end{cases},
	\label{eq:analytical_h}
	\end{equation}
where $R \equiv \sqrt{gD_0} / f_0$ is the Rossby radius. In geostrophic balance, the meridional velocity is just $v = -\partial_x h$, giving
	\begin{equation}
		v = -\frac{g h_0}{f_0R} \exp(-\abs{x} / R)
	\label{analytical_v}
	\end{equation}
for the final state.

\subsection{Energetics}
The total energy of a system with density $\rho$ can be decomposed into potential and kinetic components
	\begin{align}
		V &= \frac{1}{2}\rho g \qty(h^2 - D^2_0) \\
		K &= \frac{1}{2}\rho D_0 \qty(u^2 + v^2),
	\end{align}
and the available potential energy is obtained by subtracting the constant term.
For an initial height perturbation eq. \ref{eq:initial_h} the resulting change in available potential and kinetic energy is
	\begin{align}
		\Delta V &= - \frac{3}{2} \rho g h^2_0 R \label{eq:change_potential} \\
		\Delta K &= \frac{1}{2} \rho g h^2_0 R \label{eq:change_kinetic}.
	\end{align}
Initially, the potential energy constitutes the total energy, and so from eqs. \ref{eq:change_potential} and \ref{eq:change_kinetic} the final kinetic energy should be a 1/3 of the total energy. Likewise, the potential energy should be a 2/3 of the total energy.