\section{Method}
\label{sec:method}
We will follow the same dicretization scheme and method as done in a previous paper \citep{lab1}, with some further implementations of reduced gravity and the beta plane.

The initial condition is a gaussian centered at the equator
	\begin{equation}
		h_i = h_0\exp\qty[-((x - 0.5*L_x)/L_w)^2 - (y/L_w)^2],
	\end{equation}
where $L_i$ denotes the domain size and $L_w$ is a scaling factor.

For studying the southern border, we will assume periodic boundary conditions in the zonal direction and a open northern border. We will look at two different cases, that of the Atlantic ocean and the Baltic sea with basin parameters shown in Table \ref{tab:southernborder}.
	\begin{table}[htbp]
		\begin{tabular}{lll}
			\textbf{Region} & \textbf{L}  & \textbf{D}$'_0$  \\
			Atlantic ocean & $10^7\, \text{m}$  & $1000\, \text{m}$  \\
			Baltic sea & $10^6\, \text{m}$ & $30\, \text{m}$ 
		\end{tabular}
		\caption{Basin parameters for the Atlantic ocean and Baltic sea. $L$ denotes the domain size, while $D'_0$ is the thermocline depth.}
		\label{tab:southernborder}
	\end{table}

When looking at the equatorial wave, we use solid boundaries in all directions with domain size $L = 2 \cdot 10^7\, \text{m}$ and thermocline depth $D'_0 = 1000\, \text{m}$.

Global parameters for the simulations are presented in Table \ref{tab:parameters}
	\begin{table}[htbp]
		\begin{tabular}{ll}
			\textbf{Parameter} & \textbf{Value} \\
			$ g' $ & $ 0.04 \, \text{m}\text{s}^{-2} $ \\
			$ f_0 $ & $ 10^{-4} \, \text{s}^{-1}$ \\
			$ \beta $ & $ 2.287 \cdot 10^{-11} \, \text{ (sm) }^{-1}  $
		\end{tabular}
		\caption{Model parameters.}
		\label{tab:parameters}
	\end{table}