\section{Discussion}
\label{sec:conclusion}

The disagreement between numerical and analytical sea-surface height profiles in Figure \ref{fig:atlanticbaltic} may be due to a poorly chosen width of the initial perturbation, causing very narrow waves. However, given a rough estimate of the propagation velocity there is reasonable agreement between analytical and numerical values (Table \ref{tab:velocities}), and so the eastward propagation direction (Figures \ref{fig:atlantic} and \ref{fig:baltic}) and velocity suggests that we are looking at Kelvin waves.

We can see clear signs of equatorial Kelvin waves, as well as Rossby waves in Figures \ref{fig:initialfinal_equatorial} and \ref{fig:equatorial}. This is also observed in the real ocean by satellites, and the model catches much of the dynamics as Rossby waves transmit information westward when the Kelvin wave crashes into the eastern boundary. 

The discrepancy between numerical and analytical velocities may be due to the presence of gravity and Rossby waves contributing to some of the kinetic energy. This would also explain why the largest difference is seen in the equatorial scenario where Rossby waves are present. Some error is clearly present in the method, which is not rigorous, and may explain the magnitude difference in discrepancy between the Atlantic and Baltic sea.

Introducing vertical stratification in the model in form of a reduced gravity was essential as the barotropic Rossby radius is too large to warrant the beta plane approximation, which is a necessary component of equatorial dynamics.


\subsection{Conclusion}
In this report, we have studied the dynamics of solid boundaries in a fluid. We solved the reduced-gravity shallow water equations with a beta plane approximation for the equator. There is reasonable quantitative agreement between numerical and analytical solutions, and good qualitative agreement. Additionally, we observed Rossby waves in a simple two layer model.